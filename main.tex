\documentclass[a4paper,12pt]{article}

\usepackage[utf8]{inputenc}
\usepackage[T1]{fontenc}
\usepackage[polish]{babel}

\usepackage[left=1cm,right=1cm,top=1cm,bottom=1cm]{geometry}

\begin{document}

\section{Wstęp}

\begin{enumerate}
    \item Wstęp
    \item Zakładanie projektów
\end{enumerate}

\section{Formatowanie tekstu}

\begin{enumerate}
    \item Wygenerować dokument o co najmniej 10 stronach -- każda strona powinna mieć inny kolor
    \item Użyć tekst ,,Lorem ipsum'' w~co najmniej 6 rozmiarach i~każdy rozmiar w~3 kolorach
    \item Napisać dowolny tekst (co najmniej 5 linii) z~uwzględnieniem symboli specjalnych podanych na wykładzie
    \item Napisać dowolny tekst (co najmniej 5 linii) z uwzględnieniem pochylania i~pogrubiania fragmentów tekstu wykorzystując deklaracje oraz komendy
    \item Przygotować spis 5 dowolnych konferencji podając ich tytuł, miejsce, organizatora  oraz daty początku i końca
\end{enumerate}

Uwagi:

\begin{itemize}
    \item Wszędzie pilnować twardych spacji (a, i, o, u, w, z)
    \item Poprawnie prezentować przedziały dat
\end{itemize}

\section{Otoczenia}

\begin{enumerate}
    \item Odwzorować strukturę UwB uwzględniając władze, wydziały i~działy administracji (lista numerowana)
    \item Odwzorować podział administracyjny Polski (wszystkie województwa, przykładowe powiaty, przykładowe gminy, przykładowe miasta i wsie) (dobrać własne punktatory) {lista punktowana}
\end{enumerate}

\end{document}