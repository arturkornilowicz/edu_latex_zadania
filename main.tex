\documentclass[a4paper,12pt]{article}

\usepackage[utf8]{inputenc}
\usepackage[T1]{fontenc}
\usepackage[polish]{babel}

\usepackage[left=1cm,right=1cm,top=1cm,bottom=1.5cm]{geometry}

\pagestyle{empty}

\begin{document}

\section{Wstęp}

\begin{enumerate}
    \item Wstęp
    \item Zakładanie projektów
\end{enumerate}

\section{Formatowanie tekstu}

\begin{enumerate}
    \item Wygenerować dokument o co najmniej 10 stronach -- każda strona powinna mieć inny kolor.
    \item Użyć tekst ,,Lorem ipsum'' w~co najmniej 6 rozmiarach i~każdy rozmiar w~3 kolorach.
    \item Napisać dowolny tekst (co najmniej 5 linii) z~uwzględnieniem symboli specjalnych podanych na wykładzie.
    \item Napisać dowolny tekst (co najmniej 5 linii) z uwzględnieniem pochylania i~pogrubiania fragmentów tekstu wykorzystując deklaracje oraz komendy.
    \item Przygotować spis 5 dowolnych konferencji podając ich tytuł (w~cudzysłowach), miejsce, organizatora oraz daty początku i końca.
\end{enumerate}

Uwagi:

\begin{itemize}
    \item Wszędzie pilnować twardych spacji (a, i, o, u, w, z).
    \item Poprawnie prezentować przedziały dat.
\end{itemize}

\section{Otoczenia}

\begin{enumerate}
    \item Odwzorować strukturę UwB uwzględniając władze (tylko stanowiska -- bez nazwisk), wydziały i~działy administracji (lista numerowana).
    \item Odwzorować podział administracyjny Polski (wszystkie województwa, przykładowe powiaty, przykładowe gminy, przykładowe miasta i wsie) (lista punktowana).
    \item Napisać program (dowolny język programowania) generujący szablon dokumentu \LaTeX'owego (klasa, pakiety, początek i~koniec dokumentu). Załączyć tekst programu do dokumentu (tekst predefiniowany).
    \item Opisać swoje 3 ulubione języki programowania (description).
    \item Użyć tekst ,,Lorem ipsum'' wyjustowany, wyśrodkowany, wyrównany do lewej oraz do prawej strony.
    \item Użyć dowolny cytat ulubionego autora/autorki -- otoczyć go własnym tekstem.
    \item Zastosować listę numerowaną do zaprezentowania dowolnego zagadnienia -- wykorzystać 4 poziomy oraz własne numeracje na każdym poziomie.
    \item Zastosować listę punktowaną do zaprezentowania dowolnego zagadnienia -- wykorzystać 4 poziomy oraz własne punktatory na każdym poziomie.
\end{enumerate}

Uwagi:

\begin{itemize}
    \item Przy definiowaniu własnej numeracji i~własnych punktatorów pilnować zasięgu definicji.
\end{itemize}

\section{Klasy dokumentów}

\begin{enumerate}
    \item Przygotowanie dokumentu z~pracą licencjacką.
    \item Książka na ulubiony temat (niekoniecznie informatyczny). W~książce powinny wystąpić różne elementy \verb!\frontmatter!, \verb!\mainmatter!, \verb!\appendix! oraz \verb!\backmatter!. Uwzględnić spis treści. 
    \item List do wybranego urzędu (dane osobowe powinny być fikcyjne). Uwzględnić załączniki.
\end{enumerate}
    
\end{document}